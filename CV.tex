%%%%%%%%%%%%%%%%%%%%%%%%%%%%%%%%%%%%%%%%%
% Clean CV without external structure.tex
%%%%%%%%%%%%%%%%%%%%%%%%%%%%%%%%%%%%%%%%%

\documentclass[10pt]{article}

% ---------- PACKAGES ----------
\usepackage[a4paper, hmargin=23mm, vmargin=30mm, top=20mm]{geometry}
\usepackage{fancyhdr}
\usepackage{lastpage}
\usepackage[T1]{fontenc}
\usepackage[utf8]{inputenc}
\usepackage{xurl}       % allow line breaks in URLs
\usepackage{hyperref}   % clickable links
\usepackage{enumitem}   % custom itemize spacing
\usepackage{titlesec}   % section formatting

% ---------- COLORS ----------
\usepackage{xcolor}
\definecolor{slateblue}{rgb}{0.17,0.22,0.34}

% ---------- HEADER ----------
\renewcommand{\title}[1]{%
{\huge\color{slateblue}\textbf{#1}}\\[2pt]
\rule{\textwidth}{0.5mm}\\
}

% ---------- SECTION STYLE ----------
\titleformat{\section}{\large\bfseries\color{slateblue}}{}{0em}{}
\titleformat{\subsection}{\normalsize\bfseries\color{slateblue}}{}{0em}{}

% ---------- PAGE STYLE ----------
\fancypagestyle{plain}{\fancyhf{}\cfoot{\thepage\ of \pageref{LastPage}}}
\pagestyle{plain}
\setlength\parindent{0pt}
\setlist[itemize]{leftmargin=1.5em, itemsep=0.3em}

% ---------- DOCUMENT ----------
\begin{document}

% ---------- NAME ----------
\title{Dr. Mohamed Aziz BOUKRAA -- Curriculum Vitae}

% ---------- CONTACT INFO ----------
\parbox{0.5\textwidth}{
\begin{tabbing}
\hspace{2.5cm} \= \hspace{4cm} \= \kill
{\bf Address} \> Gaustadalléen 23B,\\
\> Ole-Johan Dahls hus,\\
\> 0373 Oslo, Norway \\
{\bf Date of Birth} \> December 11$^{th}$, 1994 \\
{\bf Website} \> \href{https://med-aziz-boukraa.github.io/}{med-aziz-boukraa.github.io}
\end{tabbing}}
\hspace{0.5cm}
\parbox{0.5\textwidth}{
\begin{tabbing}
\hspace{3.5cm} \= \hspace{4cm} \= \kill
{\bf Nationality} \> French \\
{\bf Mobile Phone} \> +33 (0)6 67 75 85 77 \\
{\bf Email (Personal)} \> \href{mailto:mohamedazizboukraa@gmail.com}{mohamedazizboukraa@gmail.com} \\
{\bf Email (Professional)} \> \href{mailto:mohambo@ifi.uio.no}{mohambo@ifi.uio.no}
\end{tabbing}}

% ---------- RESEARCH INTERESTS ----------
\section{Research Interests}
Mathematical Modeling, Inverse Problems, Wave Propagation, Seismic Imaging, Full Waveform Inversion, Numerical Analysis, Partial Differential Equations, Finite Element Method, Scientific Computing, High Performance Computing, Applications to realistic environments.


% ---------- EDUCATION ----------
\section{Education}

\textbf{2018--2021} \quad Doctorate in Applied Mathematics, Laboratory of Mathematics Nicolas Oresme (LMNO), \href{http://www.unicaen.fr/}{University of Caen Normandy, France}.  

\textbf{Title:} \textit{Méthodes inverses à régularisation évanescente pour l'identification de conditions aux limites en théorie des plaques minces}.  
\textbf{English title:} \textit{Fading regularization inverse methods for the identification of boundary conditions in thin plate theory.}  

\textbf{Thesis supervisor:} Pr. Franck Delvare. \quad \textbf{Defense:} December 14, 2021.


\textbf{2016--2018} \quad Master's degree in Mathematics, Spec. Optimization and Mathematical Physics, \href{https://www.univ-tln.fr/}{University of Toulon, France}.  

\textbf{2013--2016} \quad Bachelor's degree in Mathematics, \href{https://www.univ-st-etienne.fr/}{University Jean Monnet, Saint-Étienne, France}.  

% ---------- EXPERIENCE ----------
\section{Experience}
\subsection{Research Experience}

\textbf{Sep 2024 -- Present} \quad Postdoctoral Research Fellow,  
\href{https://www.mn.uio.no/ifi/english/}{Department of Informatics},  
\href{https://www.uio.no/}{University of Oslo}, Norway.   \textit{Research in inverse problems and image reconstruction in acoustics and elastography}. 
This position is part of the MSCA CO-FUND \href{https://www.uio.no/dscience/english/dstrain/postdoctoral-fellows/aziz-boukraa.html}{DSTrain} programme.  The project is carried out in collaboration with the  
\href{https://www.mn.uio.no/ifi/english/research/groups/dsp/}{Digital Signal Processing group (DSB), IFI, University of Oslo}. 



\textbf{Sep 2022 -- Août 2024} \quad Postdoctoral position.  
\href{https://uma.ensta-paris.fr/idefix/}{INRIA (IDEFIX Team)}, \href{https://www.ensta-paris.fr/}{Ensta Paris}, \href{https://www.ip-paris.fr/}{Institut Polytechnique de Paris}, and \href{https://www.edf.fr/}{EDF R\&D}.  
\textit{Development of inverse problem methodologies for interface imaging, application to the rock-concrete interface for a hydroelectric dam.}

\medskip

\textbf{Sep 2021 -- Août 2022} \quad Contractual assistant professor ("ATER").  
\href{https://lmbp.uca.fr/}{Laboratoire de mathématiques Blaise Pascal}, \href{http://polytech.univ-bpclermont.fr/}{Polytech Clermont INP}.  

\medskip

\textbf{Sep 2018 -- Dec 2021} \quad Doctoral thesis.  
Laboratory of Mathematics Nicolas Oresme (LMNO), University of Caen Normandy, France.  
\href{https://tel.archives-ouvertes.fr/tel-03526539}{PDF link}.

\medskip

\textbf{Mar 2018 -- Aug 2018} \quad Research internship (MSc). Alten SA, Alten Innovation Center, Chaville, France.  

\medskip

\textbf{Mar 2017 -- Jun 2017} \quad Research internship (MSc). LIS Laboratory, University of Toulon, France.  

\subsection{Teaching Experience}

\textbf{Sep 2021 -- Août 2022} \quad Polytech Clermont INP.  
Contractual assistant professor (192h).  
\begin{itemize}
  \item Tutorials in Mathematics, Probability, Statistics, and Numerical Analysis.
  \item Lectures and tutorials for Architect-Engineer double degree.
\end{itemize}

\textbf{Oct 2019 -- Sep 2021} \quad University of Caen Normandy.  
Part-time teacher (75h).  
\begin{itemize}
  \item Lectures and tutorials in Analysis, Complex Numbers, Geometry.
\end{itemize}

% ---------- SKILLS ----------
\section{Skills}
\textbf{Languages:} French (Native), Arabic (Native), English (Fluent), Spanish (Intermediate).  

\textbf{Software:} Matlab, FreeFEM, PETSc, MPI, Python, Maple, LaTeX, Gmsh.  

% ---------- INTERNATIONAL COLLABORATIONS ----------
\section*{International Collaborations}
Visiting Researcher at the National School of Engineers of Tunis -- PHC Utique Project (2020, 2022).  
Collaboration with LAMSIN laboratory on finite element algorithms with fading regularization for inverse problems.

% ---------- SCHOOLS ----------
\section*{International Schools}
\begin{itemize}
  \item Bath Symposium on Inverse Problems and AI in Medicine, University of Bath, UK, June 2025. \url{https://bathsymposium.ac.uk/symposium/inverse-problems-and-artificial-intelligence-in-medicine/}
  \item Geilo Winter School 2025 on Inverse Problems, SINTEF/NTNU, Geilo, Norway, Jan 2025. \url{https://www.sintef.no/projectweb/geilowinterschool/2025-inverse-problems/}
  \item French–Romanian Summer School on Applied Mathematics, University of Bucharest, Romania, July 2019. \url{https://sites.google.com/site/marinliviu/french-romanian-summer-school-on-applied-mathematics/6th-french-romanian-summer-school-on-applied-mathematics-3-11-july-2019}
\end{itemize}

% ---------- PUBLICATIONS ----------
\section{Publications and Communications}
\subsection{Journal Papers}
% BEGIN JOURNALS
\begin{enumerate}
\item \textbf{Boukraa, M. A.}, Delvare, F., Caill{\'e}, L.. Fading regularization method for an inverse boundary value problem associated with the biharmonic equation. \textit{Journal of Computational and Applied Mathematics, 440, 115755, 2025}.: \href{https://doi.org/10.1016/j.cam.2024.116285}{10.1016/j.cam.2024.116285}
\item \textbf{Boukraa, M. A.}, Amdouni, S., Delvare, F.. Fading regularization FEM algorithms for the Cauchy problem associated with the two-dimensional biharmonic equation. \textit{Mathematical Methods in the Applied Sciences, 46(2), 2389--2412, 2023}.: \href{https://doi.org/10.1002/mma.8651}{10.1002/mma.8651}
\end{enumerate}
% END JOURNALS

\subsection{Conference Proceedings}
% BEGIN CONFS
\begin{enumerate}
\item \textbf{Boukraa, M. A.}, Audibert, L., Bonazzoli, M., Haddar, H., Vautrin, D.. High-Resolution Seismic Imaging for Dam-Rock Interface using Full-Waveform Inversion. \textit{Waves 2024, Berlin, Germany, 2024}.: \href{https://doi.org/10.17617/3.MBE4AA}{\nolinkurl{10.17617/3.MBE4AA}}
\item \textbf{Boukraa, M. A.}, Bonazzoli, M., Haddar, H., Audibert, L., Vautrin, D.. Imaging a dam-rock interface with inversion of a full elastic-acoustic model. \textit{11th International Conference on Inverse Problems in Engineering (ICIPE 2024), Rio de Janeiro, Brazil, 2024}.: \href{https://uca.hal.science/hal-04661884}{\nolinkurl{https://uca.hal.science/hal-04661884}}
\item \textbf{Boukraa, M. A.}, Audibert, L., Bonazzoli, M., Haddar, H., Vautrin, D.. Imagerie d’interface barrage-fondation par inversion de forme d’onde complète. \textit{E3S Web of Conferences, 2024}.: \href{https://doi.org/10.1051/e3sconf/202450404002}{\nolinkurl{10.1051/e3sconf/202450404002}}
\end{enumerate}
% END CONFS

\subsection{Talks in International Conferences}
% BEGIN CONF TALKS
\begin{itemize}
\item 11th International Conference on Inverse Problems, Control and Shape Optimization (PICOF 2025), Hammamet, Tunisia, October 28--31 2025. \url{https://picof2025.sciencesconf.org/}
\item IEEE International Ultrasonics Symposium (IUS 2025), Utrecht, Netherlands, April 2025. \url{https://2025.ieee-ius.org/}
\item Ultrasonic Imaging and Tissue Characterization Conference (UITC 2025), Silver Spring, MD, USA, April 2025.
\item 48th Scandinavian Symposium on Physical Acoustics (SSPA 2025), Geilo, Norway, January 26--29 2025. \url{https://www.norskfysisk.no/faggrupper/faggruppe-akustikk/sspa/}
\item WAVES 2024 (16th International Conference on Mathematical and Numerical Aspects of Wave Propagation), Berlin, Germany, June 2024. \url{https://waves2024.mps.mpg.de/}
\item 11th International Conference on Inverse Problems in Engineering (ICIPE24), Rio de Janeiro, Brazil, June 2024. \url{https://icipe2024.org/}
\item 11th International Conference on Inverse Problems in odeling and Simulations (IPMS 2024), Malta, May 2024. \url{https://www.ipms-conference.org/ipms2024/}
\item 11th Applied Inverse Problems Conference (AIP 2023), Göttingen, Germany, September 2023. \url{https://aip2023.de/}
\item 10th International Conference on Inverse Problems, Control and Shape Optimization (PICOF'22), Caen, France, October 25--27 2022. \url{https://picof22.sciencesconf.org/}
\item 10th International Conference on Inverse Problems in Engineering (ICIPE20), Francavilla al Mare (Chieti), Italy, May 15--19 2022. \url{https://iopscience.iop.org/article/10.1088/1742-6596/2444/1/011001}
\end{itemize}
% END CONF TALKS

\subsection{Other Talks}
% BEGIN OTHER TALKS
\begin{itemize}
\item Journées Scientifiques AGAP Qualité Géophysique Appliquée, Poitiers, France, March 26--28 2024. \url{https://www.agapqualite.org/2023/07/11/journees-scientifiques-agap-26-au-28-mars-2024/}
\item Congrès des Jeunes Chercheurs en Mécanique (Méca-J 2023), Online, August 28--30 2023. \url{https://mecaj2023.sciencesconf.org/}
\item 11ème Biennale de la Société de Mathématiques Appliquées et Industrielles (SMAI 2023), Le Gosier, Guadeloupe, France, May 22--26 2023. \url{https://smai2023.math.cnrs.fr/fr/}
\item Congrès Français de Mécanique (CFM 2022), Nantes, France, August 29 -- September 2 2022. \url{https://cfm2022.fr/}
\item 45ème Congrès National d'Analyse Numérique (CANUM 2022), Évian-les-Bains, France, June 13--17 2022. \url{https://canum2020.math.cnrs.fr}
\item Congrès des Jeunes Chercheurs en Mécanique (Méca-J 2021), Online, August 25--27 2021. \url{http://meca-j.sciencesconf.org}
\item 10ème Biennale de la Société de Mathématiques Appliquées et Industrielles (SMAI 2021), La Grande-Motte, France, June 21--25 2021. \url{https://smai2021.math.univ-toulouse.fr}
\item 9ème Biennale de la Société de Mathématiques Appliquées et Industrielles (SMAI 2019), Guidel-Plages (Morbihan), France, May 13--17 2019. \url{http://smai.emath.fr/smai2019/index.php}
\end{itemize}
% END OTHER TALKS


\section{Activities of Collective Interest}

\begin{itemize}
  \item Participation at \textit{"Semaines d'Etudes Mathématiques et Entreprises" (SEME)},  
  Pointe-à-Pitre, Guadeloupe, May 15–19, 2023.  
  \href{https://semeantilles.sciencesconf.org/}{semeantilles.sciencesconf.org}  
  Joint Work: "Optimal distribution for steam production".
  Industry Collaboration: Refinery Industry SARA (Pôle 972, Martinique).

  \item Organizing committee member of the international conference \textbf{PICOF’22},  
  Caen, October 25–27, 2022.  
  \href{https://picof22.sciencesconf.org/}{picof22.sciencesconf.org}

  \item Organization of the workshop \textit{“Chercheurs-Chercheuses”},  
  Dôme de Caen, February 2, 2020.
\end{itemize}



\end{document}
