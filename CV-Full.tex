%----------------------------------------------------------------------------------------
%	PACKAGES AND OTHER DOCUMENT CONFIGURATIONS
%----------------------------------------------------------------------------------------

\documentclass[10pt]{article} % Default font size

%%%%%%%%%%%%%%%%%%%%%%%%%%%%%%%%%%%%%%%%%
% Wilson Resume/CV
% Structure Specification File
% Version 1.0 (22/1/2015)
%
% This file has been downloaded from:
% http://www.LaTeXTemplates.com
%
% License:
% CC BY-NC-SA 3.0 (http://creativecommons.org/licenses/by-nc-sa/3.0/)
%
%%%%%%%%%%%%%%%%%%%%%%%%%%%%%%%%%%%%%%%%%

%----------------------------------------------------------------------------------------
%	PACKAGES AND OTHER DOCUMENT CONFIGURATIONS
%----------------------------------------------------------------------------------------

\usepackage[a4paper, hmargin=23mm, vmargin=30mm, top=20mm]{geometry} % Use A4 paper and set margins

\usepackage{fancyhdr} % Customize the header and footer

\usepackage{lastpage} % Required for calculating the number of pages in the document

\usepackage{hyperref} % Colors for links, text and headings

\setcounter{secnumdepth}{0} % Suppress section numbering

%\usepackage[proportional,scaled=1.064]{erewhon} % Use the Erewhon font
%\usepackage[erewhon,vvarbb,bigdelims]{newtxmath} % Use the Erewhon font
\usepackage[utf8]{inputenc} % Required for inputting international characters
\usepackage[T1]{fontenc} % Output font encoding for international characters

\usepackage{fontspec} % Required for specification of custom fonts
\setmainfont[Path = ./fonts/,
Extension = .otf,
BoldFont = Erewhon-Bold,
ItalicFont = Erewhon-Italic,
BoldItalicFont = Erewhon-BoldItalic,
SmallCapsFeatures = {Letters = SmallCaps}
]{Erewhon-Regular}

\usepackage{color} % Required for custom colors
\definecolor{slateblue}{rgb}{0.17,0.22,0.34}

\usepackage{sectsty} % Allows customization of titles
\sectionfont{\color{slateblue}} % Color section titles

\fancypagestyle{plain}{\fancyhf{}\cfoot{\thepage\ of \pageref{LastPage}}} % Define a custom page style
\pagestyle{plain} % Use the custom page style through the document
\renewcommand{\headrulewidth}{0pt} % Disable the default header rule
\renewcommand{\footrulewidth}{0pt} % Disable the default footer rule

\setlength\parindent{0pt} % Stop paragraph indentation

% Non-indenting itemize
\newenvironment{itemize-noindent}
{\setlength{\leftmargini}{0em}\begin{itemize}}
{\end{itemize}}

% Text width for tabbing environments
\newlength{\smallertextwidth}
\setlength{\smallertextwidth}{\textwidth}
\addtolength{\smallertextwidth}{-2cm}

\newcommand{\sqbullet}{~\vrule height 1ex width .8ex depth -.2ex} % Custom square bullet point definition

%----------------------------------------------------------------------------------------
%	MAIN HEADER COMMAND
%----------------------------------------------------------------------------------------

\renewcommand{\title}[1]{
{\huge{\color{slateblue}\textbf{#1}}}\\ % Header section name and color
\rule{\textwidth}{0.5mm}\\ % Rule under the header
}

%----------------------------------------------------------------------------------------
%	JOB COMMAND
%----------------------------------------------------------------------------------------

\newcommand{\job}[6]{
\begin{tabbing}
\hspace{2cm} \= \kill
\textbf{#1} \> \href{#4}{#3} \\
\textbf{#2} \>\+ \textit{#5} \\
\begin{minipage}{\smallertextwidth}
\vspace{2mm}
#6
\end{minipage}
\end{tabbing}
\vspace{2mm}
}

%----------------------------------------------------------------------------------------
%	SKILL GROUP COMMAND
%----------------------------------------------------------------------------------------

\newcommand{\skillgroup}[2]{
\begin{tabbing}
\hspace{5mm} \= \kill
\sqbullet \>\+ \textbf{#1} \\
\begin{minipage}{\smallertextwidth}
\vspace{2mm}
#2
\end{minipage}
\end{tabbing}
}

%----------------------------------------------------------------------------------------
%	INTERESTS GROUP COMMAND
%-----------------------------------------------------------------------------------------

\newcommand{\interestsgroup}[1]{
\begin{tabbing}
\hspace{5mm} \= \kill
#1
\end{tabbing}
\vspace{-10mm}
}

\newcommand{\interest}[1]{\sqbullet \> \textbf{#1}\\[3pt]} % Define a custom command for individual interests

%----------------------------------------------------------------------------------------
%	TABBED BLOCK COMMAND
%----------------------------------------------------------------------------------------

\newcommand{\tabbedblock}[1]{
\begin{tabbing}
\hspace{2cm} \= \hspace{4cm} \= \kill
#1
\end{tabbing}
} % Include the file specifying document layout

\usepackage{enumitem}

\usepackage{enumitem}

% Force all itemize lists to wrap at a narrower width (e.g. 0.9\textwidth)
\setlist[itemize]{leftmargin=*, labelsep=0.5em, align=parleft, before=\setlength{\linewidth}{0.9\textwidth}}

\begin{document}

%----------------------------------------------------------------------------------------
%	NAME AND CONTACT INFORMATION
%----------------------------------------------------------------------------------------

\title{Dr. Mohamed Aziz BOUKRAA -- Curriculum Vitae} % Print the main header




\parbox{0.5\textwidth}{ % First block
\begin{tabbing} % Enables tabbing
\hspace{2.5cm} \= \hspace{4cm} \= \kill % Spacing within the block
{\bf Address} \> Gaustadalléen 23B,\\ % Adresse line 1 
\> Ole-Johan Dahls hus,\\
\> 0373 Oslo, Norway \\ % Adresse line 2
{\bf Date of Birth} \>  December 11$^{th}$, 1994 \\% Date of birth 
\end{tabbing}}
%\hfill % Horizontal space between the two blocks
\hspace{0.5cm}
\parbox{0.5\textwidth}{ % Second block
\begin{tabbing} % Enables tabbing
\hspace{3.5cm} \= \hspace{4cm} \= \kill % Spacing within the block
{\bf Nationality} \> French \\% Nationality
{\bf Mobile Phone} \> +33 (0)6 67 75 85 77 \\ % Mobile phone
{\bf Email (Personal)} \> \href{mailto:mohamedazizboukraa@gmail.com}{mohamedazizboukraa@gmail.com} \\ % Email address
{\bf Email (Professional)} \> \href{mailto:mohambo@ifi.uio.no}{mohambo@ifi.uio.no} \\ % Email address
\end{tabbing}}

\vspace{-0.7cm}
{\bf Website}  \href{https://med-aziz-boukraa.github.io/}{https://med-aziz-boukraa.github.io/} \\ % Email address
%----------------------------------------------------------------------------------------
%	PERSONAL PROFILE
%----------------------------------------------------------------------------------------

\section{Research Interests}

Mathematical Modeling, Inverse Problems, Wave Propagation,  Seismic Imaging, Full Waveform Inversion,  Numerical Analysis, Partial Differential Equations,  Finite Element Method, Scientific Computing, High Performance Computing, 
 Applications to realistic environments.
%Structural Mechanics,
%Parallel and Distributed Computing, Optimization,  Wave Propagation, Imaging,  Parallel and Distributed Computing, Applications to realistic environments, Non-destructive testing.


%Recently graduated from the University of Caen Normandy with a doctorate in Applied Mathematics, I carried out several research internships in laboratories and R\&D companies. I developed a sharp research expertise in the field of inverse problems, numerical analysis, mathematical modeling and implementation of numerical methods. A confident lecturer and classroom teacher, able to explain complex information to audiences of all levels.


%----------------------------------------------------------------------------------------
%	EDUCATION SECTION
%----------------------------------------------------------------------------------------

\section{Education}

\tabbedblock{
\bf{2018-2021} \> \textbf{Doctorate in Applied Mathematics}, Laboratory of Mathematics Nicolas Oresme (LMNO), \\ \>  \href{http://www.unicaen.fr/}{University of Caen Normandy, France} \\[5pt]
\>\+ \textbf{Title} : \textit{«Méthodes inverses à régularisation évanescente pour l'identification de conditions}\>\\\textit{aux limites en théorie des plaques minces»}\>\\
\textbf{English Title} : «\textit{Fading regularization inverse methods for the identification of boundary}\\ \textit{conditions in thin plate theory»} \> \\
\textbf{Thesis supervisor} : Pr. \textit{Franck Delvare} \> \\
Defended: \underline{December 14, 2021}\> \\
\textbf{Comity members} :\\
\textit{-M. Mejdi AZAIEZ, Professeur des universités, Institut Polytechnique de Bordeaux}\\
\textit{-M. Leonardo BAFFICO, Maître de conférences, Université de Caen}\\
\textit{-Mme. Amel BEN ABDA, Professeur des universités, ENIT de Tunis}\\
\textit{-Mme. Laëtitia CAILLÉ, Maître de conférences, Université de Poitiers}\\
\textit{-M. Fabien CAUBET, Maître de conférences (HDR), Université de Pau et des Pays de l’Adour}\\
\textit{-Mme. Juliette LEBLOND, Directeur de recherche, INRIA Centre de Recherche Sophia Antipolis}\\
\textit{-M. Liviu MARIN, Professeur des universités, Université de Bucarest}
}

%------------------------------------------------

\tabbedblock{
\bf{2016-2018} \> \textbf{Master's degree in Mathematics, Spec. Optimization and Mathematical Physics} \\ \>  \href{https://www.univ-tln.fr/}{University of Toulon, France}
%\>\+
%\textit{Pure Mathematics} \> A \\
%\textit{Statistics (AS)} \> A \\
%\textit{Physics} \> A \\
%\textit{Economics} \> B
}

\tabbedblock{
	\bf{2013-2016} \> \textbf{Bachelor's degree in Mathematics} \\ \>  \href{https://www.univ-st-etienne.fr/}{University Jean Monnet, Saint-Étienne, France} 
%	\>First Class - 80\% Average\\
%	\>\+
%	\textit{Third Year Project - 89\% awarded `Project of the Year 2007'}
}

%\tabbedblock{
	%\bf{2013} \> \textbf{High school diploma in Mathematics} \\ \>  Carthage Hannibal High school, Tunis, Tunisia}
	
	
%	\>First Class - 80\% Average\\
%	\>\+
%	\textit{Third Year Project - 89\% awarded `Project of the Year 2007'}
%}

%----------------------------------------------------------------------------------------
%	EMPLOYMENT HISTORY SECTION
%----------------------------------------------------------------------------------------

\section{Experience}
\subsection{Research Experience}
%------------------------------------------------
\job
{Sep 2022 -}{Août 2024}
{\textbf{Postdoctoral position}}
{https://uma.ensta-paris.fr/idefix/}
{\href{https://uma.ensta-paris.fr/idefix/}{INRIA (IDEFIX Team)}, \href{https://www.ensta-paris.fr/}{Ensta Paris},  \href{https://www.ip-paris.fr/}{Institut Polytechnique de Paris} and \href{https://www.edf.fr/groupe-edf/inventer-l-avenir-de-l-energie/r-d-un-savoir-faire-mondial/5-ans-edf-lab-saclay/edf-lab-chatou-75-ans-de-recherche }{EDF R\&D}}
{«\textbf{\textit{Development of inverse problem methodologies for interface imaging, application to the rock-concrete interface for a hydroelectric dam}}»}
{The research project is carried out within the IDEFIX team of \textit{INRIA Paris Saclay} center. It consists of a collaboration between \textit{EDF  R\&D} (Electricity of France), \textit{INRIA} and the \textit{Institut Polytechnique de Paris}. The objective is to work on interface imaging problems arising in electricity production sites.
The main case study is the imaging of the interface between the concrete of a dam and the rock on which it is built. To deal with this application, we developed a quantitative reconstruction method based on  full-waveform inversion (FWI) techniques using non-destructive seismic waves generated from the dam's walls. The aim is then to couple this approach with sampling methods for this complex geometric configuration. The goal is to achieve this for two types of non-destructive testing, mechanical waves which correspond to ultrasonic measurements and electromagnetic waves which correspond to RADAR measurements. The analysis takes into account both real and simulated measurements, with the aim of optimizing future measurement campaigns.
}

%\newpage
%------------------------------------------------
\job
{Sep 2021 -}{Août 2022}
{\textbf{Contractual assistant professor ("ATER" contract)}}
{https://lmbp.uca.fr/}
{\href{http://polytech.univ-bpclermont.fr/}{Polytech Clermont INP (département IMDS)}) \&   \href{https://lmbp.uca.fr/}{Laboratoire de mathématiques Blaise Pascal  (LMBP)}}
{\href{http://polytech.univ-bpclermont.fr/}{http://polytech.univ-bpclermont.fr/}     \&   \href{https://lmbp.uca.fr/}{https://lmbp.uca.fr/}  }
{ The position is a one-year full-time job,  associated with both the \textit{Mathematical Engineering and Data Science (IMDS)} department of the Polytech Clermont INP engineering school for the teaching part and the \textit{laboratory of mathematics Blaise Pascal}  for the research part.
	This contract started 3 months before my thesis defense. It made it possible to cover this period of the end of the thesis and also to finalize the articles resulting from my thesis. Research activities have also been carried out such as participation in seminars and conferences as well as research missions abroad.
	}


%------------------------------------------------
\job
{Sep 2018 -}{Dec 2021}
{\textbf{Doctoral thesis} (\href{https://tel.archives-ouvertes.fr/tel-03526539}{PDF : https://tel.archives-ouvertes.fr/tel-03526539})}
{https://tel.archives-ouvertes.fr/tel-03526539}
{Laboratory of Mathematics Nicolas Oresme (LMNO), University of Caen Normandy, France}
{«\textbf{\textit{Fading regularization inverse methods for the identification of boundary conditions in thin plate theory}}»}
{
My thesis, started in September 2018 under the supervision of Pr. Franck Delvare, at the University of Caen Normandy,
was defended on December 14, 2021.

In this thesis, we investigate the solution of the Cauchy problem associated with the biharmonic equation using the fading regularization method proposed by A. Cimetière, F. Delvare, M. Jaoua, and F. Pons (2000-2001). A particular attention is devoted to the numerical implementation of the iterative algorithm used for the resolution and this by using various numerical methods such as the method of fundamental solutions and the finite element method while proposing a new stopping criterion of the iterative algorithm. Once the mathematical problem, with mathematical boundary conditions, has been treated, the study thus carried out is then generalized to the study of the Cauchy problem in thin plate theory, since in mechanics, the bending of thin plates under Kirchhoff-Love’s assumptions, is governed by the same differential equation. From a numerical point of view, plate finite elements based on Kirchhoff theory known as discrete Kirchhoff finite elements are combined with the fading regularization technique to solve the problem. This strategy made it possible to obtain accurate reconstructions of the solution and of its normal derivative on the whole boundary, in particular for non-smooth geometries. The results thus obtained were an inspiration for the use of plate-type finite elements to solve Cauchy problems associated with second order partial differential equations. The obtained results are very competitive with those of previous studies. Robustness against high level noisy data is also an advantage of this strategy.
}



\job
{Mar 2018 -}{Aug 2018}
{\textbf{Research internship} (2$^{nd}$ year Master's degree research project)}
{http://www.alten.fr}
{Alten SA, Alten Innovation Center, Chaville (île-de-France 92)}
{«\textbf{\textit{Modeling the performance of an image reconstruction device for images degraded by raindrops}}»}

{	As part of the second year of my Master's degree research project, I chose to do a research internship within the company Alten in the Parisian region. The subject is part of the theme of driving assistance systems that Alten is constantly developing in its R\&D centers. The research carried out concerns the design of a system of stereoscopic cameras, embedded on cars, whose role is to reconstruct the image of the scene in the event that it is degraded by raindrops. The established strategy is to design performance models based on the intrinsic characteristics of the cameras, the geometric characteristics of the device and the characteristics of the drops, which will allow to evaluate the performance of each parameter on the reconstruction capacity of the scene. This internship combines both theoretical and numerical modeling skills as well as the ability to build a performance model with a probabilistic aspect.

}


%------------------------------------------------

\job
{Mar 2017 -}{Jun 2017}
{\textbf{Research internship} (1$^{st}$ year Master's degree research project)}
{https://www.lis-lab.fr/}
{Laboratoire LIS, SeaTech, University of Toulon}
{«\textbf{\textit{Observability of linear control systems and proof of convergence of the deterministic Kalman filter}}»}


{	The internship consists of a research project during my first year of my Master's degree. It was carried out at the \textit{LIS} laboratory of the \textit{SeaTech} engineering school of the University of Toulon. The subject enters the field of automatic and control systems, where we studied in particular the observability of linear control systems by taking the Kalman filter as an example of an observer. We finally established its proof of convergence in a deterministic framework as well as the numerical validation of this result.
	
}

\subsection{Teaching Experience}
\job
{Sep 2021 -}{Août 2022}
{\textbf{Polytech Clermont INP, Université de Clermont Auvergne}}
{http://polytech.univ-bpclermont.fr/}
{Contractual assistant professor (one-year full-time job: 192h)}
{
	\begin{itemize-noindent}
		\item{	\textbf{Tutorials for first year in engineering cycle (common block)}}
		\begin{itemize-noindent}
			\item{ Maths TC (3 groups$\times$18h) : Matrix reduction, Jordan diagonalization, Solving differential systems, Laplace transformation,... }
			\item{ Probability et Statistics (2 groups$\times$12h) : Discrete Laws, Continuous Laws, Estimates, Confidence Interval, Hypothesis Testing,...}
         \item{ Numerical analysis (3 groups$\times$14h): Methods of solving linear systems, Interpolation and approximation of functions, Numerical solving of differential equations,...}
		\end{itemize-noindent}
    \item{ \textbf{Lectures and Tutorials, double degree course Architect-Engineer}}
		\begin{itemize-noindent}
\item{Level Bac+3 (1 group: 14h Lecture + 16h Tutorial): Matrix reduction, Jordan diagonalization, Reduction of quadratic forms, Conics, Solving differential systems, Double integrals,... }
\item{Level Bac+1 (1 group: 14h Lecture + 16h Tutorial): Study of functions, Derivatives and primitives, Differential equations, Integral calculation,... }
		\end{itemize-noindent}	
\end{itemize-noindent}
}

%------------------------------------------------
\job
{Oct 2019 -}{Sep 2021}
{\textbf{University  of Caen Normandy}}
{http://www.unicaen.fr}
{Part-time teacher, amendment to the doctoral contract (25h+50h).}
{\begin{itemize-noindent}
\item{\textbf{Lectures and Tutorials of Calculation Techniques for first year of Bachelor degree} (Teaching of general mathematics: Analysis, Complex Numbers, Plane and Space Geometry,...)}
	\begin{itemize-noindent}
\item{2020-2021: 2 groups $\times$ 25h (Bachelor in Mechanics).}
\item{2019-2020: 1 group $\times$ 25h (Bachelor in common block Maths-Mechanics)}
	\end{itemize-noindent}
\end{itemize-noindent}}

%------------------------------------------------









%----------------------------------------------------------------------------------------
%	IT/COMPUTING SKILLS SECTION
%----------------------------------------------------------------------------------------
\section{Skills}
\parbox{0.5\textwidth}{ % First block
	\begin{tabbing} % Enables tabbing
		\hspace{2.5cm} \= \hspace{1.5cm}\= \hspace{4cm} \= \kill % Spacing within the block
		{\bf Linguistic} \> \textit{French} \>:   Native\\ % Address line 1
		\> 	\textit{Arabic} \>:  Native\\
		\> 	\textit{English} \>:  Fluent\\
		\> 	\textit{Spanish} \>:   Intermediate\\
\end{tabbing}}
%\hfill % Horizontal space between the two blocks
\hspace{0.5cm}
\parbox{0.5\textwidth}{ % Second block
	\begin{tabbing} % Enables tabbing
		\hspace{3.5cm} \= \hspace{2cm} \= \kill % Spacing within the block
		{\bf Software} 	\>  \textit{Matlab}\\
\> \textit{FreeFEM} - \textit{PETSc} - \textit{MPI}\\
\> \textit{Python}\\
\> \textit{Maple}\\
\> 	\textit{Latex}\\
\> 	\textit{GMsh} 
\end{tabbing}}


%------------------------------------------------
\section*{International Collaborations}

\skillgroup{Visiting Researcher at the National School of Engineers of Tunis - PHC Utique Project}
{
\textbf{Duration:} 07/01-31/2020 \& 07/01-31/2022

\textbf{Description:} The research mission is part of the PHC Utique project. It consists in collaboration with Tunisian researchers from LAMSIN laboratory. This collaboration occurred twice, first during my second year of thesis and later during my ATER contract. Our joint efforts were focused on the development of finite element algorithms with fading regularization method for solving inverse problems. While significant progress was made during these visits, there are ongoing opportunities for further research and development in this area.
}


%------------------------------------------------
\section*{International Schools}

\begin{itemize}

  \item \textbf{Bath Symposium on Inverse Problems and Artificial Intelligence in Medicine},  
  University of Bath, UK, June 2025.  
  \href{https://bathsymposium.ac.uk/symposium/inverse-problems-and-artificial-intelligence-in-medicine/}{bathsymposium.ac.uk}

  \item \textbf{Geilo Winter School 2025 on Inverse Problems},  
  SINTEF / NTNU, Geilo, Norway, January 2025.  
  \href{https://www.sintef.no/projectweb/geilowinterschool/2025-inverse-problems/}{sintef.no/geilowinterschool/2025-inverse-problems}


  \item \textbf{French–Romanian Summer School on Applied Mathematics},  
  University of Bucharest, Sinaia, Romania, July 3–11, 2019.  
  \href{https://sites.google.com/site/marinliviu/french-romanian-summer-school-on-applied-mathematics/6th-french-romanian-summer-school-on-applied-mathematics-3-11-july-2019}{sites.google.com/marinliviu/summer-school-2019}
\end{itemize}

\section{Publications and communications}
%------------------------------------------------
\subsection{Journal papers}
% BEGIN JOURNALS
\begin{enumerate}
\item \textbf{Boukraa, M. A.}, Delvare, F., Caill{\'e}, L.. Fading regularization method for an inverse boundary value problem associated with the biharmonic equation. \textit{Journal of Computational and Applied Mathematics, 440, 115755, 2025}.: \href{https://doi.org/10.1016/j.cam.2024.116285}{10.1016/j.cam.2024.116285}
\item \textbf{Boukraa, M. A.}, Amdouni, S., Delvare, F.. Fading regularization FEM algorithms for the Cauchy problem associated with the two-dimensional biharmonic equation. \textit{Mathematical Methods in the Applied Sciences, 46(2), 2389--2412, 2023}.: \href{https://doi.org/10.1002/mma.8651}{10.1002/mma.8651}
\end{enumerate}
% END JOURNALS

\subsection{Conference proceedings with peer review}
% BEGIN CONFS
\begin{enumerate}
\item \textbf{Boukraa, M. A.}, Audibert, L., Bonazzoli, M., Haddar, H., Vautrin, D.. High-Resolution Seismic Imaging for Dam-Rock Interface using Full-Waveform Inversion. \textit{Waves 2024, Berlin, Germany, 2024}.: \href{https://doi.org/10.17617/3.MBE4AA}{\nolinkurl{10.17617/3.MBE4AA}}
\item \textbf{Boukraa, M. A.}, Bonazzoli, M., Haddar, H., Audibert, L., Vautrin, D.. Imaging a dam-rock interface with inversion of a full elastic-acoustic model. \textit{11th International Conference on Inverse Problems in Engineering (ICIPE 2024), Rio de Janeiro, Brazil, 2024}.: \href{https://uca.hal.science/hal-04661884}{\nolinkurl{https://uca.hal.science/hal-04661884}}
\item \textbf{Boukraa, M. A.}, Audibert, L., Bonazzoli, M., Haddar, H., Vautrin, D.. Imagerie d’interface barrage-fondation par inversion de forme d’onde complète. \textit{E3S Web of Conferences, 2024}.: \href{https://doi.org/10.1051/e3sconf/202450404002}{\nolinkurl{10.1051/e3sconf/202450404002}}
\end{enumerate}
% END CONFS




%------------------------------------------------
\subsection{Talks in International Conferences}
% BEGIN CONF TALKS
\begin{itemize}
\item 11th International Conference on Inverse Problems, Control and Shape Optimization (PICOF 2025), Hammamet, Tunisia, October 28--31 2025. \url{https://picof2025.sciencesconf.org/}
\item IEEE International Ultrasonics Symposium (IUS 2025), Utrecht, Netherlands, April 2025. \url{https://2025.ieee-ius.org/}
\item Ultrasonic Imaging and Tissue Characterization Conference (UITC 2025), Silver Spring, MD, USA, April 2025.
\item 48th Scandinavian Symposium on Physical Acoustics (SSPA 2025), Geilo, Norway, January 26--29 2025. \url{https://www.norskfysisk.no/faggrupper/faggruppe-akustikk/sspa/}
\item WAVES 2024 (16th International Conference on Mathematical and Numerical Aspects of Wave Propagation), Berlin, Germany, June 2024. \url{https://waves2024.mps.mpg.de/}
\item 11th International Conference on Inverse Problems in Engineering (ICIPE24), Rio de Janeiro, Brazil, June 2024. \url{https://icipe2024.org/}
\item 11th International Conference on Inverse Problems in odeling and Simulations (IPMS 2024), Malta, May 2024. \url{https://www.ipms-conference.org/ipms2024/}
\item 11th Applied Inverse Problems Conference (AIP 2023), Göttingen, Germany, September 2023. \url{https://aip2023.de/}
\item 10th International Conference on Inverse Problems, Control and Shape Optimization (PICOF'22), Caen, France, October 25--27 2022. \url{https://picof22.sciencesconf.org/}
\item 10th International Conference on Inverse Problems in Engineering (ICIPE20), Francavilla al Mare (Chieti), Italy, May 15--19 2022. \url{https://iopscience.iop.org/article/10.1088/1742-6596/2444/1/011001}
\end{itemize}
% END CONF TALKS

\subsection{Other Talks}
% BEGIN OTHER TALKS
\begin{itemize}
\item Journées Scientifiques AGAP Qualité Géophysique Appliquée, Poitiers, France, March 26--28 2024. \url{https://www.agapqualite.org/2023/07/11/journees-scientifiques-agap-26-au-28-mars-2024/}
\item Congrès des Jeunes Chercheurs en Mécanique (Méca-J 2023), Online, August 28--30 2023. \url{https://mecaj2023.sciencesconf.org/}
\item 11ème Biennale de la Société de Mathématiques Appliquées et Industrielles (SMAI 2023), Le Gosier, Guadeloupe, France, May 22--26 2023. \url{https://smai2023.math.cnrs.fr/fr/}
\item Congrès Français de Mécanique (CFM 2022), Nantes, France, August 29 -- September 2 2022. \url{https://cfm2022.fr/}
\item 45ème Congrès National d'Analyse Numérique (CANUM 2022), Évian-les-Bains, France, June 13--17 2022. \url{https://canum2020.math.cnrs.fr}
\item Congrès des Jeunes Chercheurs en Mécanique (Méca-J 2021), Online, August 25--27 2021. \url{http://meca-j.sciencesconf.org}
\item 10ème Biennale de la Société de Mathématiques Appliquées et Industrielles (SMAI 2021), La Grande-Motte, France, June 21--25 2021. \url{https://smai2021.math.univ-toulouse.fr}
\item 9ème Biennale de la Société de Mathématiques Appliquées et Industrielles (SMAI 2019), Guidel-Plages (Morbihan), France, May 13--17 2019. \url{http://smai.emath.fr/smai2019/index.php}
\end{itemize}
% END OTHER TALKS

\section{Activities of Collective Interest}

\begin{itemize}
  \item Participation at \textit{"Semaines d'Etudes Mathématiques et Entreprises" (SEME)},  
  Pointe-à-Pitre, Guadeloupe, May 15–19, 2023.  
  \href{https://semeantilles.sciencesconf.org/}{semeantilles.sciencesconf.org}  
  Joint Work: "Optimal distribution for steam production".
  Industry Collaboration: Refinery Industry SARA (Pôle 972, Martinique).

  \item Organizing committee member of the international conference \textbf{PICOF’22},  
  Caen, October 25–27, 2022.  
  \href{https://picof22.sciencesconf.org/}{picof22.sciencesconf.org}

  \item Organization of the workshop \textit{“Chercheurs-Chercheuses”},  
  Dôme de Caen, February 2, 2020.
\end{itemize}


\end{document}